\documentclass[12pt]{article}%
%\usepackage{titlesec}
%\setcounter{secnumdepth}{4}

%\titleformat{\paragraph}
%{\normalfont\normalsize\bfseries}{\theparagraph}{1em}{}
%\titlespacing*{\paragraph}
%{0pt}{3.25ex plus 1ex minus .2ex}{1.5ex plus .2ex}
\usepackage{titlesec}
\usepackage{hyperref}

\titleclass{\subsubsubsection}{straight}[\subsection]

\newcounter{subsubsubsection}[subsubsection]
\renewcommand\thesubsubsubsection{\thesubsubsection.\arabic{subsubsubsection}}
\renewcommand\theparagraph{\thesubsubsubsection.\arabic{paragraph}} % optional; useful if paragraphs are to be numbered

\titleformat{\subsubsubsection}
  {\normalfont\normalsize\bfseries}{\thesubsubsubsection}{1em}{}
\titlespacing*{\subsubsubsection}
{0pt}{3.25ex plus 1ex minus .2ex}{1.5ex plus .2ex}

\makeatletter
\renewcommand\paragraph{\@startsection{paragraph}{5}{\z@}%
  {3.25ex \@plus1ex \@minus.2ex}%
  {-1em}%
  {\normalfont\normalsize\bfseries}}
\renewcommand\subparagraph{\@startsection{subparagraph}{6}{\parindent}%
  {3.25ex \@plus1ex \@minus .2ex}%
  {-1em}%
  {\normalfont\normalsize\bfseries}}
\def\toclevel@subsubsubsection{4}
\def\toclevel@paragraph{5}
\def\toclevel@paragraph{6}
\def\l@subsubsubsection{\@dottedtocline{4}{7em}{4em}}
\def\l@paragraph{\@dottedtocline{5}{10em}{5em}}
\def\l@subparagraph{\@dottedtocline{6}{14em}{6em}}
\makeatother

\setcounter{secnumdepth}{4}
\setcounter{tocdepth}{4}


\begin{titlepage}
	\clearpage\thispagestyle{empty}
	\centering
	\vspace{2cm}

	% Titles
	%{\large  \par}
	%\vspace{4cm}
	{\Huge \textbf{Software Design Document \\
        for \\
        Employee Record Database}} \\
	\vspace{1cm}
	{\large \textbf{Programming Assignment 1 \\
      January 29, 2020} \par}
	\vspace{4cm}
	{\normalsize Prepared By \\ % \\ specifies a new line
      Paul Abers \\
      pa0034@uah.edu \par}
	\vspace{2cm}

    \vspace{2cm}

	% Information about the University
	{\normalsize Prepared For \\
		Mr. James Williamson \\
		CS 221, Data Structures in C++ \\
        Computer Science Department \\
        University of Alabama in Huntsville \par}

      \vspace{2cm}

	\pagebreak

\end{titlepage}

\tableofcontents

\section{Overview}
The purpose of this assignment is to provide a simple and easy way
to access an employee record class. The class must store an employee ID,
employee name, department and annual salary of each employee. The class
will eventually be ported to an employee database.

\section{Referenced Documents}
Programming Assignment 1 Statement of Work.

\section{Architectural Design}
\subsection{Concept of Execution}
This program creates a class structure to store information for an
individual employee. The class stores an employee's first and last name,
a unique employee ID, the employee's department ID and the employee's salary.

A database manager will have access to public get and set methods of the class
in order to set the various attributes for the employee as well as get them later.
There is also a default constructor that initializes the class as well as a constructor
that handles all inputs being included. A quick and easy print will also be provided for quickly
displaying all attributes of the class.

\subsection{Abstract Data Type}
The employee record structure is implemented with a class structure separated in a cpp and header file.

\subsection{Code Outline}
This program will consist of the following fiels: EmployeeRecord.h and EmployeeRecord.cpp
\begin{itemize}
    \item EmployeeRecord() -- default constructor
    \item EmplyoeeRecord(int ID, char *fName, char *lName, int dept, double sal) -- initialization constructor
    \item getID() -- return int value of employee id
    \item setID(int ID) -- set employee id
    \item getName(char* fName, char *lName) -- copy employee's first and last name into pointers passed
    \item setName(char* fName, char *lName) -- set employee's first and last name to pointers passed
    \item getDept(int& d) -- get value of employee's department
    \item setDept(int d) -- set value of employee's department
    \item getSalary(double *sal) -- pointer function to get employee's salary
    \item setSalary(double sal) -- set employee's salary

\end{itemize}

\section{Detailed Design}
\subsection{Source File: EmployeeRecord.h and EmployeeRecord.cpp}
\subsubsection{Function: main()}
\subsubsubsection{Purpose}
\subsubsubsection{Arguments}
\subsubsubsection{Return Value}
\subsubsubsection{Function Outline in Pseudocode}
\subsubsubsection{Tracability}

\subsubsection{Function: main()}
\subsubsubsection{Purpose}
\subsubsubsection{Arguments}
\subsubsubsection{Return Value}
\subsubsubsection{Function Outline in Pseudocode}
\subsubsubsection{Tracability}

\subsubsection{Function: main()}
  \subsubsubsection{Purpose}
\subsubsubsection{Arguments}
\subsubsubsection{Return Value}
\subsubsubsection{Function Outline in Pseudocode}
\subsubsubsection{Tracability}


\end{document}