\documentclass[12pt]{article}%

\usepackage{titlesec}
\usepackage{hyperref}

\titleclass{\subsubsubsection}{straight}[\subsection]

\newcounter{subsubsubsection}[subsubsection]
\renewcommand\thesubsubsubsection{\thesubsubsection.\arabic{subsubsubsection}}
\renewcommand\theparagraph{\thesubsubsubsection.\arabic{paragraph}} % optional; useful if paragraphs are to be numbered

\titleformat{\subsubsubsection}
  {\normalfont\normalsize\bfseries}{\thesubsubsubsection}{1em}{}
\titlespacing*{\subsubsubsection}
{0pt}{3.25ex plus 1ex minus .2ex}{1.5ex plus .2ex}

\makeatletter
\renewcommand\paragraph{\@startsection{paragraph}{5}{\z@}%
  {3.25ex \@plus1ex \@minus.2ex}%
  {-1em}%
  {\normalfont\normalsize\bfseries}}
\renewcommand\subparagraph{\@startsection{subparagraph}{6}{\parindent}%
  {3.25ex \@plus1ex \@minus .2ex}%
  {-1em}%
  {\normalfont\normalsize\bfseries}}
\def\toclevel@subsubsubsection{4}
\def\toclevel@paragraph{5}
\def\toclevel@paragraph{6}
\def\l@subsubsubsection{\@dottedtocline{4}{7em}{4em}}
\def\l@paragraph{\@dottedtocline{5}{10em}{5em}}
\def\l@subparagraph{\@dottedtocline{6}{14em}{6em}}
\makeatother

\setcounter{secnumdepth}{4}
\setcounter{tocdepth}{4}

\begin{document}

\begin{titlepage}
    \clearpage\thispagestyle{empty}
    \centering
    \vspace{2cm}

    % Titles
    %{\large  \par}
    %\vspace{4cm}
    {\Huge \textbf{Software Design Document \\
        for \\
        Employee Record, Store, and Customer Database}} \\
    \vspace{1cm}
    {\large \textbf{Programming Assignment 3 \\
      March 16, 2020} \par}
    \vspace{4cm}
    {\normalsize Prepared By \\ % \\ specifies a new line
      Paul Abers \\
      pa0034@uah.edu \par}
    \vspace{2cm}

    \vspace{2cm}

    % Information about the University
    {\normalsize Prepared For \\
        Mr. James Williamson \\
        CS 221, Data Structures in C++ \\
        Computer Science Department \\
        University of Alabama in Huntsville \par}

      \vspace{2cm}

	\pagebreak

\end{titlepage}

\tableofcontents

\section{System Overview}
The purpose of this assignment is to provide a simple and easy way
to access an employee record database. The database has four parts,
the employee record class, a store class, a customer list class and
an employee record database.
The employee record class must store an employee ID,
employee name, department, annual salary of each employee and a customer
list class. The customer list class stores a list of store classes. The store
class stores information on a single customer store. The employee database will
use a binary tree to store employee records.

\section{Referenced Documents}
Programming Assignment 1 Statement of Work. \\
Programming Assignment 2 Statement of Work. \\
Programming Assignment 3 Statement of Work.

\section{Architectural Design}
\subsection{Concept of Execution}
This program creates a class structure to store information for an
individual employee, a store, a list of stores and a binary search tree of employee records.
The employee class stores an employee's first and last name, a unique employee ID, the employee's department ID, the employee's salary, a pointer to a left and right child, and a list of customer stores for the employee. The customer list class stores a list of customer stores.

A database manager will have access to public get and set methods of the class
in order to set the various attributes for the employee as well as get them later.
There is also a get customer list function.
There is also a default constructor that initializes the class as well as a constructor
that handles all inputs being included. A quick and easy print will also be provided for quickly
displaying all attributes of the class.
The customer list class will have functions
for adding a store, removing a store, getting a store based on store id and printing a store info.
The employee database will have a pointer to the root of the tree and a pointer to an ifstream object.

\subsection{Abstract Data Type}
The employee record structure is implemented with a class structure separated in a cpp and header file. Likewise for the store class. The customer list is a linked list of store object. The employee database is a binary search tree.

\subsection{Code Outline}
This program will consist of the following files: EmployeeRecord.h, EmployeeRecord.cpp, Store.h, Store.cpp,
CustomerList.h, CustomerList.cpp, EmployeeDatabase.h and EmployeeDatabase.cpp.
\hfill\\
\hfill\\
EmployeeRecord Class
\underline{Private Attributes}:
\begin{itemize}
    \item m\_iEmployeeID -- int value for employee id
    \item m\_sLastName -- character array of length 32 for last name
    \item m\_sFirstName -- character array of length 32 for first name
    \item m\_iDeptId -- int for department id
    \item m\_dSalary -- double for employee's salary
    \item m\_pCustomerList -- pointer to customer list object
    \item m\_pLeft -- pointer to left child
    \item m\_pRight -- pointer to right child
    \end{itemize}
    \hfill\\
\underline{Public Methods}:
\begin{itemize}
    \item EmployeeRecord() -- default constructor
    \item EmplyoeeRecord() -- initialization constructor
    \item getID() -- return int value of employee id
    \item setID() -- set employee id
    \item getName() -- copy employee's first and last name into pointers passed
    \item setName() -- set employee's first and last name to pointers passed
    \item getDept() -- get value of employee's department
    \item setDept() -- set value of employee's department
    \item getSalary() -- pointer function to get employee's salary
    \item setSalary() -- set employee's salary
    \item printRecord() -- prints to screen all data for employee's record
    \item getCustomerList() -- return the pointer to the employee record's customer list object
    \item removeCustomerList() -- set pointer to customer list to Null
    \item destroyCustomerList() -- deletes the customer list objects and sets pointer to null
\end{itemize}
\hfill\\
\hfill\\
Store Class
\underline{Private Attributes}:
\begin{itemize}
    \item m\_iStoreID -- pointer integer for store ID
    \item m\_sStoreName -- pointer to character array of size 64 for store name
    \item m\_sAdress -- pointer to character array of size 64 for address
    \item m\_sCity -- pointer to character array of size 32 for city
    \item m\_sState -- pointer to character array of size 32 for state
    \item m\_sZip -- pointer to character array of size 11 for zipcode
\end{itemize}
    \hfill\\
\underline{Public Methods}:
\begin{itemize}
    \item m\_pNext -- pointer to next store for CustomerList object
    \item Store() -- constructor for Store object
    \item ~Store() -- destructor for Store object
    \item getStoreID() -- return integer for store ID
    \item setStoreID() -- set integer for Store ID
    \item getStoreName() -- get character array for store name
    \item setStoreName() -- set character array for store name
    \item getStoreAddress() -- get character array for store address
    \item setStoreAddress() -- set character array for store address
    \item getStoreCity() -- get character array for store city
    \item setStoreCity() -- set character array for store city
    \item getStoreState() -- get character array for store state
    \item setStoreState() -- set character array for store state
    \item getStoreZip() -- get character array for store zipcode
    \item setStoreZip() -- set character array for store zipcode
    \item printStoreInfo() -- print all the stores info
\end{itemize}
\hfill\\
\hfill\\
CustomerList Class
\underline{Private Attributes}:
\begin{itemize}
    \item m\_pHead -- pointer to first store in list
    \end{itemize}
    \hfill\\
\underline{Public Methods}:
\begin{itemize}
    \item CustomerList() -- constructor for store
    \item ~CustomerList() -- destructor for store
    \item addStore() -- add a store to list
    \item removeStore() -- remove a store from list
    \item getStore() -- access a store from list
    \item updateStore() -- update a store in list
    \item printStoresInfo() -- print info for each store in list
    \end{itemize}
\hfill\\
\hfill\\
EmployeeDatabase Class
\underline{Private Attributes}:
\begin{itemize}
    \item m\_pRoot -- pointer to root employee record object
    \item inFile -- ifstream object
    \end{itemize}
    \hfill\\
\underline{Public Methods}:
\begin{itemize}
    \item EmployeeDatabase() -- constructor for employee database
    \item ~EmployeeDatabase() -- destructor for employee database
    \item addEmployee() -- add a employee to tree
    \item removeEmployee() -- remove an employee from tree
    \item getEmployee() -- access an employee from tree
    \item printEmployeeRecords -- print employee records of tree
    \item buildDatabase -- build database from an input file
\end{itemize}

\section{Detailed Design}
\subsection{Source File: EmployeeRecord.h and EmployeeRecord.cpp}
\subsubsection{Function: EmployeeRecord()}
\subsubsubsection{Purpose}
This is the default constructor for the EmployeeRecord class.
\subsubsubsection{Arguments}
This default constructor takes no arguments.
\subsubsubsection{Return Value}
A constructor, therefore no value is returned.
\subsubsubsection{Function Outline in Pseudocode}
\begin{verbatim}
Set employee id to 0
set last name to ""
set first name to ""
set department id to 0
set salary to 0.0
set left child to null
set right child to null
\end{verbatim}
\subsubsubsection{Tracability}
This function will fulfil requirement 2.2.2.1 of SOW

\subsubsection{Function: EmplyoeeRecord()}
\subsubsubsection{Purpose}
This is the optional constructor to set all values passed into function.
\subsubsubsection{Arguments}
int employee id, character array pointer first name, character array pointer last name, int for
department id, double for salary.
\subsubsubsection{Return Value}
None
\subsubsubsection{Function Outline in Pseudocode}
\begin{verbatim}
Set employee id to ID
copy passed character array for lName into m_sLastName
copy passed character array for fName into m_sFirstName
set department id to dept
set salary to sal.
set left child to null
set right child to null
\end{verbatim}
\subsubsubsection{Tracability}
This function will fulfil requirement 2.2.2.2 of SOW

\subsubsection{Function: \textasciitilde EmplyoeeRecord()}
\subsubsubsection{Purpose}
This is the destructor for the employee record.
\subsubsubsection{Arguments}
None
\subsubsubsection{Return Value}
None
\subsubsubsection{Function Outline in Pseudocode}
\begin{verbatim}
Properly destruct the class. Clean up and deallocate memory initialized for pointers for the first and last name
character arrays. Properly destruct customer list.
\end{verbatim}
\subsubsubsection{Tracability}
This function will fulfil requirement 2.2.2.3 of SOW

\subsubsection{Function:  getID()}
\subsubsubsection{Purpose}
This function allows a user to get the private employee ID.
\subsubsubsection{Arguments}
None
\subsubsubsection{Return Value}
Int value stored for employee id.
\subsubsubsection{Function Outline in Pseudocode}
\begin{verbatim}
Return value of member stored employee ID.
\end{verbatim}
\subsubsubsection{Tracability}
This function will partially fulfil requirement 2.2.2.4 of SOW

\subsubsection{Function:  setID()}
\subsubsubsection{Purpose}
This function allows a user to set the private employee ID.
\subsubsubsection{Arguments}
Int value to set the member stored employee id to.
\subsubsubsection{Return Value}
Void
\subsubsubsection{Function Outline in Pseudocode}
\begin{verbatim}
Set internal member variable for employee id to passed integer value.
return
\end{verbatim}
\subsubsubsection{Tracability}
This function will partially fulfil requirement 2.2.2.4 of SOW


\subsubsection{Function:  getName()}
\subsubsubsection{Purpose}
This function allows a user to get the private employee first and last names.
\subsubsubsection{Arguments}
Pointer to character array first name, pointer to character array last name
\subsubsubsection{Return Value}
void
\subsubsubsection{Function Outline in Pseudocode}
\begin{verbatim}
Copy contents of internal member variable character arrays for first and last name into the
character arrays passed into the function.
return
\end{verbatim}
\subsubsubsection{Tracability}
This function will partially fulfil requirement 2.2.2.5 of SOW


\subsubsection{Function:  setName()}
\subsubsubsection{Purpose}
This function allows a user to set the private employee first name and last name character arrays.
\subsubsubsection{Arguments}
Pointer to character array first name, pointer to character array last name
\subsubsubsection{Return Value}
void
\subsubsubsection{Function Outline in Pseudocode}
\begin{verbatim}
copy passed character array for lName into m_sLastName
copy passed character array for fName into m_sFirstName
return
\end{verbatim}
\subsubsubsection{Tracability}
This function will partially fulfil requirement 2.2.2.5 of SOW

\subsubsection{Function:  getDept()}
\subsubsubsection{Purpose}
This function allows a user to get the internal member value for department id.
\subsubsubsection{Arguments}
noneint reference variable
\subsubsubsection{Return Value}
void
\subsubsubsection{Function Outline in Pseudocode}
\begin{verbatim}
return m_iDeptID
\end{verbatim}
\subsubsubsection{Tracability}
This function will partially fulfil requirement 2.2.2.6 of SOW and 2.0.2.4 of SOW2

\subsubsection{Function:  setDept()}
\subsubsubsection{Purpose}
This function allows a user to set the employee department id.
\subsubsubsection{Arguments}
Int for department id
\subsubsubsection{Return Value}
void
\subsubsubsection{Function Outline in Pseudocode}
\begin{verbatim}
Set internal value for department id equal to the passed int value.
\end{verbatim}
\subsubsubsection{Tracability}
This function will partially fulfil requirement 2.2.2.6 of SOW

\subsubsection{Function:  getSalary()}
\subsubsubsection{Purpose}
Get the employee's salary.
\subsubsubsection{Arguments}
none
\subsubsubsection{Return Value}
void
\subsubsubsection{Function Outline in Pseudocode}
\begin{verbatim}
return m_dSalary
\end{verbatim}
\subsubsubsection{Tracability}
This function will partially fulfil requirement 2.2.2.7 of SOW

\subsubsection{Function:  setSalary()}
\subsubsubsection{Purpose}
This function allows a user to set the employee's salary.
\subsubsubsection{Arguments}
double for salary
\subsubsubsection{Return Value}
void
\subsubsubsection{Function Outline in Pseudocode}
\begin{verbatim}
Set member variable for salary equal to the passed double variable.
return
\end{verbatim}
\subsubsubsection{Tracability}
This function will partially fulfil requirement 2.2.2.7 of SOW

\subsubsection{Function:  printRecord()}
\subsubsubsection{Purpose}
This function prints all info for employee record to the screen.
\subsubsubsection{Arguments}
None
\subsubsubsection{Return Value}
void
\subsubsubsection{Function Outline in Pseudocode}
\begin{verbatim}
Print header for employee record
print employee id
print employee first name
print employee last name
print employee department id
print employee salary
return
\end{verbatim}
\subsubsubsection{Tracability}
This function will fulfil requirement 2.2.2.8 of SOW

\subsubsection{Function:  removeCustomerList()}
\subsubsubsection{Purpose}
This function removes the customer list from employee record and sets pointer to null
\subsubsubsection{Arguments}
None
\subsubsubsection{Return Value}
void
\subsubsubsection{Function Outline in Pseudocode}
\begin{verbatim}
check if pointer to customer is null
if not null
    set pointer to customer list to null
return
\end{verbatim}
\subsubsubsection{Tracability}
This function will fulfil requirement 2.2.2 of SOW 3

\subsubsection{Function:  destroyCustomerList()}
\subsubsubsection{Purpose}
This function deletes the customer list from employee record and sets pointer to null
\subsubsubsection{Arguments}
None
\subsubsubsection{Return Value}
void
\subsubsubsection{Function Outline in Pseudocode}
\begin{verbatim}
check if pointer to customer is null
if not null
    destroy the customer object
    set pointer to customer list to null
return
\end{verbatim}
\subsubsubsection{Tracability}
This function will fulfil requirement 2.2.2 of SOW 3

\subsection{Source File: CustomerList.h and CustomerList.cpp}
\subsubsection{Function: CustomerList()}
\subsubsubsection{Purpose}
This is the default constructor for the CustomerList class.
\subsubsubsection{Arguments}
This default constructor takes no arguments.
\subsubsubsection{Return Value}
A constructor, therefore no value is returned.
\subsubsubsection{Function Outline in Pseudocode}
\begin{verbatim}
set m_pHead to Null
\end{verbatim}
\subsubsubsection{Tracability}
This function will fulfil requirement 2.2.2.1 of SOW by providing a default constructor for the
employee record class.

\subsubsection{Function: \textasciitilde CustomerList()}
\subsubsubsection{Purpose}
This is the default destructor for the CustomerList class.
\subsubsubsection{Arguments}
This default constructor takes no arguments.
\subsubsubsection{Return Value}
A constructor, therefore no value is returned.
\subsubsubsection{Function Outline in Pseudocode}
\begin{verbatim}
set curr equal to m_pHead
iterate through list until curr's next store id is null
    set next equal to curr next pointer
    delete curr
    set curr equal to next
delete curr
return
\end{verbatim}
Start at m\_pHead \\
Loop over each Store object in linked list \\
delete each Store object
\subsubsubsection{Tracability}
This function was not specified in Statement of Work.

\subsubsection{Function: addStore()}
\subsubsubsection{Purpose}
Add a store to the customer list.
\subsubsubsection{Arguments}
A pointer to a store object.
\subsubsubsection{Return Value}
Bool to indicate success of insertion.
\subsubsubsection{Function Outline in Pseudocode}
\begin{verbatim}
set success equal to false
set curr equal to m_pHead
Iterate through list until curr's next store id is null
    if curr next store id is greater than input store id
        if curr id is less than input store id
            set input store next pointer equal to curr next pointer
            set curr next pointer equal to input store
            set success equal to true
            break loop
return success
\end{verbatim}
Set value for m\_pNext of Store object equal to store object pointer argument \\
return true
\subsubsubsection{Tracability}
This function will fulfil requirement 2.0.4.2.1 of SOW 2.

\subsubsection{Function: removeStore()}
\subsubsubsection{Purpose}
Remove a store from the customer list.
\subsubsubsection{Arguments}
Integer for store with id ID to remove
\subsubsubsection{Return Value}
A pointer to a store object.
\subsubsubsection{Function Outline in Pseudocode}
\begin{verbatim}
set temp equal to null
set curr equal to m_pHead
Iterate through list until curr's next store equals null
    if curr's next store id equals input id
        set temp equal to curr next store
        set curr next store equal to temp next store
        set temp next store equal to null
        break loop
return temp
\end{verbatim}
Start at m\_pHead \\
Iterate through list until m\_pHead's next store's id equals input id. \\
Set temporary store variable equal to m\_pHead's next store \\
Set m\_pHead's next pointer equal to m\_pHead's next next pointer\\
Set temporary store variables next pointer equal to NULL\\
break loop iteration \\
return temporary store.
\subsubsubsection{Tracability}
This function will fulfil requirement 2.0.4.2.2 of SOW2

\subsubsection{Function: getStore()}
\subsubsubsection{Purpose}
Return a pointer to a store object with a given Id if in the list.
\subsubsubsection{Arguments}
Integer for store ID to get.
\subsubsubsection{Return Value}
a pointer to a store object if Id found, else NULL
\subsubsubsection{Function Outline in Pseudocode}
\begin{verbatim}
set curr equal to m_pHead
Iterate through list until curr's next store equals null
    if curr id equals input id
        break loop
return curr
\end{verbatim}
\subsubsubsection{Tracability}
This function will fulfil requirement 2.0.4.2.3 of SOW2

\subsubsection{Function: updateStore()}
\subsubsubsection{Purpose}
Update a stores value
\subsubsubsection{Arguments}
Integer for store Id to update, char array for name of store, char array for address of store,
char array for city of store, char aray for street of store, char array for zipcode of store.
\subsubsubsection{Return Value}
A boolean to indicate success of insertion.
\subsubsubsection{Function Outline in Pseudocode}
\begin{verbatim}
Set success equal to false
Set found equal to false
Set curr equal to m_pHead
Iterate through list until curr's next store equals input id
    Set found equal to true
    break loop
If found is true
    advance curr to next store
    update all the store's value
    set success equal to true
return success
\end{verbatim}
set success equal to false \\
Start at m\_pHead \\
Iterate through list until m\_pHead's next store's id equals input id. \\
\quad If m\_pHead's next store id equals input id, call all set functions for next store data with input args \\
\quad \quad set success equal to true \\
\quad \quad break loop \\
return success
\subsubsubsection{Tracability}
This function will fulfil requirement 2.0.4.2.4 of SOW2

\subsubsection{Function: printStoresInfo()}
\subsubsubsection{Purpose}
Print all store info for all stores in customer list
\subsubsubsection{Arguments}
No arguments
\subsubsubsection{Return Value}
void
\subsubsubsection{Function Outline in Pseudocode}
\begin{verbatim}
Start loop at m_pHead
Loop over each store in customer list
    print store info
return
\end{verbatim}
\subsubsubsection{Tracability}
This function will fulfil requirement 2.0.4.2.5 of SOW2

\subsection{Source File: EmployeeDatabase.h and EmployeeDatabase.cpp}
\subsubsection{Function: EmployeeDatabase()}
\subsubsubsection{Purpose}
This is the default constructor for the EmployeeDatabase class.
\subsubsubsection{Arguments}
This default constructor takes no arguments.
\subsubsubsection{Return Value}
A constructor, therefore no value is returned.
\subsubsubsection{Function Outline in Pseudocode}
\begin{verbatim}
Set m_pRoot to NULL
set inFile to NULL
\end{verbatim}
\subsubsubsection{Tracability}
This function will fulfil requirement 2.3.2.1 of SOW 3

\subsubsection{Function: \textasciitilde EmployeeDatabase()}
\subsubsubsection{Purpose}
This is the default destructor for the EmployeeDatabase class.
\subsubsubsection{Arguments}
This default destructor takes no arguments.
\subsubsubsection{Return Value}
A destructor, therefore no value is returned.
\subsubsubsection{Function Outline in Pseudocode}
\begin{verbatim}
If root is not null
    next left is root left child
    next right is root right child
    delete(next left)
    delete(next right)
delete(root)
return
\end{verbatim}
\subsubsubsection{Tracability}
This function will fulfil requirement 2.3.2.1 of SOW 3

\subsubsection{Function: addEmployee()}
\subsubsubsection{Purpose}
Add an employee to the EmployeeDatabase tree.
\subsubsubsection{Arguments}
pointer to an EmployeeRecord object
\subsubsubsection{Return Value}
bool for success of insertion
\subsubsubsection{Function Outline in Pseudocode}
\begin{verbatim}
if root is null
    root equals employee record
else
    addEmployeeHelp(root, employeerecord)
return true
\end{verbatim}
\subsubsubsection{Tracability}
This function will partially fulfil requirement 2.3.2.2 of SOW 3

\subsubsection{Function: addEmployeeHelp()}
\subsubsubsection{Purpose}
Add an employee to the EmployeeDatabase tree.
\subsubsubsection{Arguments}
pointer to current node, pointer to an EmployeeRecord object
\subsubsubsection{Return Value}
void
\subsubsubsection{Function Outline in Pseudocode}
\begin{verbatim}
if node employee record is greater than employee record id
    if node left child is null
        node left child equals employee record
    else
        addEmployeeHelp(node left child, employee record)
elif node employee record is less than employee record id
    if node right child is null
        node right child equals employee record
    else
        addEmployeeHelp(node right child, employee record)
else
    return false
return true
\end{verbatim}
\subsubsubsection{Tracability}
This function will partially fulfil requirement 2.3.2.2 of SOW 3

\subsubsection{Function: getEmployee()}
\subsubsubsection{Purpose}
get an employee from the EmployeeDatabase tree.
\subsubsubsection{Arguments}
integer id of employee to get
\subsubsubsection{Return Value}
pointer to employee record object found
\subsubsubsection{Function Outline in Pseudocode}
\begin{verbatim}
return getEmployeeHelp(root, id)
\end{verbatim}
\subsubsubsection{Tracability}
This function will partially fulfil requirement 2.3.2.3 of SOW 3

\subsubsection{Function: getEmployeeHelp()}
\subsubsubsection{Purpose}
get an employee from the EmployeeDatabase tree helper function.
\subsubsubsection{Arguments}
current node to search at, integer id of employee to get
\subsubsubsection{Return Value}
pointer to employee record object found
\subsubsubsection{Function Outline in Pseudocode}
\begin{verbatim}
if node is null
    return null
elif node id equals id
    return node
elif node employee record id is greater than  id
    return getEmployeeHelp(node left child, id)
elif node employee record id is less than id
    return getEmployeeHelp(node right child, id)
else
    return null
\end{verbatim}
\subsubsubsection{Tracability}
This function will partially fulfil requirement 2.3.2.3 of SOW 3

\subsubsection{Function: removeEmployee()}
\subsubsubsection{Purpose}
remove an employee from the EmployeeDatabase tree.
\subsubsubsection{Arguments}
integer id of employee to remove
\subsubsubsection{Return Value}
pointer to employee record object removed
\subsubsubsection{Function Outline in Pseudocode}
\begin{verbatim}
if root is null
    return null
elif root id equals id
    if root left child and root right child are null
        temp equals root
        root equals null
        return temp
    elif root left child is null
        temp equals root
        root equals root right child
        return temp
    elif root right child is null
        temp equals root
        root equals root left child
        return temp
else
    return removeEmployeeHelp(root, id)
\end{verbatim}
\subsubsubsection{Tracability}
This function will partially fulfil requirement 2.3.2.3 of SOW 3

\subsubsection{Function: removeEmployeeHelp()}
\subsubsubsection{Purpose}
remove an employee from the EmployeeDatabase helper function.
\subsubsubsection{Arguments}
node to check at, integer id of employee to remove
\subsubsubsection{Return Value}
pointer to employee record removed
\subsubsubsection{Function Outline in Pseudocode}
\begin{verbatim}
if node id is greater than id
    if node left child id equals id
        return handleRemoveEmployee(node, id)
    elif node left child is null
        return null
    else
        return removeEmployeeHelp(node left child)
if node id is less than id
    if node right child id equals id
        return handleRemoveEmployee(node, id)
    elif node right child is null
        return null
    else
        return removeEmployeeHelp(node left child)

\end{verbatim}
\subsubsubsection{Tracability}
This function will partially fulfil requirement 2.3.2.4 of SOW 3

\subsubsection{Function: handleRemoveEmployee()}
\subsubsubsection{Purpose}
remove an employee from the EmployeeDatabase handle function.
\subsubsubsection{Arguments}
node to remove
\subsubsubsection{Return Value}
pointer to employee record removed
\subsubsubsection{Function Outline in Pseudocode}
\begin{verbatim}
if node left child id equals id
    temp equals node left child
    if temp left child and right child are null
        node left child equals temp left child
    elif temp left child is null
        node left child equals temp right child
    elif temp right child is null
        node left child equals temp left child
    else
        predecessor equals getPredecessor(temp, temp left child)
        node left child equals predecessor
else
    temp equals node right child
    if temp left child and right child are null
        node right child equals temp right child
    elif temp left child is null
        node right child equals temp right child
    elif temp right child is null
        node right child equals temp left child
    else
        predecessor equals getPredecessor(temp, temp left child)
        node right child equals predecessor
return temp
\end{verbatim}
\subsubsubsection{Tracability}
This function will partially fulfil requirement 2.3.2.4 of SOW 3

\subsubsection{Function: getPredecessor()}
\subsubsubsection{Purpose}
get predecessor for node to remove
\subsubsubsection{Arguments}
node to remove, node to remove traverser
\subsubsubsection{Return Value}
pointer to employee record predecessor
\subsubsubsection{Function Outline in Pseudocode}
\begin{verbatim}
if node traverser right child is not null
    return getpredecessor(node, node traverser right child)
else
    if node traverser left child is not null
        node right child equals node traverser left child
    return node traverser
\end{verbatim}
\subsubsubsection{Tracability}
This function will partially fulfil requirement 2.3.2.4 of SOW 3

\subsubsection{Function: printEmployeeRecords()}
\subsubsubsection{Purpose}
print employee records database public function
\subsubsubsection{Arguments}
no arguments
\subsubsubsection{Return Value}
void
\subsubsubsection{Function Outline in Pseudocode}
\begin{verbatim}
    printEmployeeRecords(root)
\end{verbatim}
\subsubsubsection{Tracability}
This function will fulfil requirement 2.3.2.5 of SOW 3

\subsubsection{Function: printEmployeeRecords()}
\subsubsubsection{Purpose}
print employee records database private function
\subsubsubsection{Arguments}
node to print
\subsubsubsection{Return Value}
void
\subsubsubsection{Function Outline in Pseudocode}
\begin{verbatim}
if node is null
    return
printEmployeeRecords(node left child)
node.printEmployeeRecord()
printEmployeeRecords(node right child)
return
\end{verbatim}
\subsubsubsection{Tracability}
This function will fulfil requirement 2.3.3.1 of SOW 3

\subsubsection{Function: buildDatabase()}
\subsubsubsection{Purpose}
build database
\subsubsubsection{Arguments}
character array for input data file
\subsubsubsection{Return Value}
bool for success of build
\subsubsubsection{Function Outline in Pseudocode}
\begin{verbatim}
Code is provided
\end{verbatim}
\subsubsubsection{Tracability}
This function will fulfil requirement 2.3.2.6 of SOW 3

\subsubsection{Function: destroyTree()}
\subsubsubsection{Purpose}
recursively traverse the tree and delete all nodes
\subsubsubsection{Arguments}
root of tree
\subsubsubsection{Return Value}
void
\subsubsubsection{Function Outline in Pseudocode}
\begin{verbatim}
if node is null
    return
destroyTree(node left child)
destroyTree(node right child)
delete node
return
\end{verbatim}
\subsubsubsection{Tracability}
This function will partially fulfil requirement 2.3.3.2 of SOW 3

\end{document}
%%% Local Variables:
%%% mode: latex
%%% TeX-master: t
%%% End:
