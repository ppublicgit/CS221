\documentclass[12pt]{article}%

\usepackage{titlesec}
\usepackage{hyperref}

\titleclass{\subsubsubsection}{straight}[\subsection]

\newcounter{subsubsubsection}[subsubsection]
\renewcommand\thesubsubsubsection{\thesubsubsection.\arabic{subsubsubsection}}
\renewcommand\theparagraph{\thesubsubsubsection.\arabic{paragraph}} % optional; useful if paragraphs are to be numbered

\titleformat{\subsubsubsection}
  {\normalfont\normalsize\bfseries}{\thesubsubsubsection}{1em}{}
\titlespacing*{\subsubsubsection}
{0pt}{3.25ex plus 1ex minus .2ex}{1.5ex plus .2ex}

\makeatletter
\renewcommand\paragraph{\@startsection{paragraph}{5}{\z@}%
  {3.25ex \@plus1ex \@minus.2ex}%
  {-1em}%
  {\normalfont\normalsize\bfseries}}
\renewcommand\subparagraph{\@startsection{subparagraph}{6}{\parindent}%
  {3.25ex \@plus1ex \@minus .2ex}%
  {-1em}%
  {\normalfont\normalsize\bfseries}}
\def\toclevel@subsubsubsection{4}
\def\toclevel@paragraph{5}
\def\toclevel@paragraph{6}
\def\l@subsubsubsection{\@dottedtocline{4}{7em}{4em}}
\def\l@paragraph{\@dottedtocline{5}{10em}{5em}}
\def\l@subparagraph{\@dottedtocline{6}{14em}{6em}}
\makeatother

\setcounter{secnumdepth}{4}
\setcounter{tocdepth}{4}

\begin{document}

\begin{titlepage}
	\clearpage\thispagestyle{empty}
	\centering
	\vspace{2cm}

	% Titles
	%{\large  \par}
	%\vspace{4cm}
	{\Huge \textbf{Software Design Document \\
        for \\
        Employee Record Database}} \\
	\vspace{1cm}
	{\large \textbf{Programming Assignment 1 \\
      January 29, 2020} \par}
	\vspace{4cm}
	{\normalsize Prepared By \\ % \\ specifies a new line
      Paul Abers \\
      pa0034@uah.edu \par}
	\vspace{2cm}

    \vspace{2cm}

	% Information about the University
	{\normalsize Prepared For \\
		Mr. James Williamson \\
		CS 221, Data Structures in C++ \\
        Computer Science Department \\
        University of Alabama in Huntsville \par}

      \vspace{2cm}

	\pagebreak

\end{titlepage}

\tableofcontents

\section{System Overview}
The purpose of this assignment is to provide a simple and easy way
to access an employee record class. The class must store an employee ID,
employee name, department and annual salary of each employee. The class
will eventually be ported to an employee database.

\section{Referenced Documents}
Programming Assignment 1 Statement of Work.

\section{Architectural Design}
\subsection{Concept of Execution}
This program creates a class structure to store information for an
individual employee. The class stores an employee's first and last name,
a unique employee ID, the employee's department ID and the employee's salary.

A database manager will have access to public get and set methods of the class
in order to set the various attributes for the employee as well as get them later.
There is also a default constructor that initializes the class as well as a constructor
that handles all inputs being included. A quick and easy print will also be provided for quickly
displaying all attributes of the class.

\subsection{Abstract Data Type}
The employee record structure is implemented with a class structure separated in a cpp and header file.

\subsection{Code Outline}
This program will consist of the following fiels: EmployeeRecord.h and EmployeeRecord.cpp.
\newline
\newline
\underline{Private Attributes}:
\begin{itemize}
    \item m\_iEmployeeID -- int value for employee id
    \item m\_sLastName -- character array of length 32 for last name
    \item m\_sFirstName -- character array of length 32 for first name
    \item m\_iDeptId -- int for department id
    \item m\_dSalary -- double for employee's salary
\end{itemize}
\newline
\newline
\underline{Public Methods}:
\newline
\begin{itemize}
    \item EmployeeRecord() -- default constructor
    \item EmplyoeeRecord(int ID, char *fName, char *lName, int dept, double sal) -- initialization constructor
    \item getID() -- return int value of employee id
    \item setID(int ID) -- set employee id
    \item getName(char* fName, char *lName) -- copy employee's first and last name into pointers passed
    \item setName(char* fName, char *lName) -- set employee's first and last name to pointers passed
    \item getDept(int\& d) -- get value of employee's department
    \item setDept(int d) -- set value of employee's department
    \item getSalary(double *sal) -- pointer function to get employee's salary
    \item setSalary(double sal) -- set employee's salary
    \item printRecord() -- prints to screen all data for employee's record
\end{itemize}

\section{Detailed Design}
\subsection{Source File: EmployeeRecord.h and EmployeeRecord.cpp}
\subsubsection{Function: EmployeeRecord()}
\subsubsubsection{Purpose}
This is the default constructor for the EmployeeRecord class.
\subsubsubsection{Arguments}
This default constructor takes no arguments.
\subsubsubsection{Return Value}
A constructor, therefore no value is returned.
\subsubsubsection{Function Outline in Pseudocode}
Set employee id to 0, last name to "", first name to "", department id to 0 and salary to 0.0.
\subsubsubsection{Tracability}
This function will fulfil requirement 2.2.2.1 of SOW by providing a default constructor for the
employee record class.

\subsubsection{Function: EmplyoeeRecord()}
\subsubsubsection{Purpose}
This is the optional constructor to set all values passed into function.
\subsubsubsection{Arguments}
int employee id, character array pointer first name, character array pointer last name, int for
department id, double for salary.
\subsubsubsection{Return Value}
None
\subsubsubsection{Function Outline in Pseudocode}
Set employee id to ID, copy passed character array for lName into m_sLastName, copy passed character array for fName into m_sFirstName, set department id to dept and set salary to sal.
\subsubsubsection{Tracability}
This function will fulfil requirement 2.2.2.2 of SOW by providing a default constructor for the
employee record class.

\subsubsection{Function: ~EmplyoeeRecord()}
\subsubsubsection{Purpose}
This is the destructor for the employee record.
\subsubsubsection{Arguments}
None
\subsubsubsection{Return Value}
None
\subsubsubsection{Function Outline in Pseudocode}
Properly destruct the class. Clean up and deallocate memory initialized for pointers for the first and last name
character arrays.
\subsubsubsection{Tracability}
This function will fulfil requirement 2.2.2.3 of SOW by providing a default constructor for the
employee record class.

\subsubsection{Function:  getID()}
\subsubsubsection{Purpose}
This function allows a user to get the private employee ID.
\subsubsubsection{Arguments}
None
\subsubsubsection{Return Value}
Int value stored for employee id.
\subsubsubsection{Function Outline in Pseudocode}
Return value of member stored employee ID.
\subsubsubsection{Tracability}
This function will partially fulfil requirement 2.2.2.4 of SOW by providing a default constructor for the
employee record class.

\subsubsection{Function:  setID()}
\subsubsubsection{Purpose}
This function allows a user to set the private employee ID.
\subsubsubsection{Arguments}
Int value to set the member stored employee id to.
\subsubsubsection{Return Value}
Void
\subsubsubsection{Function Outline in Pseudocode}
Set internal member variable for employee id to passed integer value.
\subsubsubsection{Tracability}
This function will partially fulfil requirement 2.2.2.4 of SOW by providing a default constructor for the
employee record class.


\subsubsection{Function:  getName()}
\subsubsubsection{Purpose}
This function allows a user to get the private employee first and last names.
\subsubsubsection{Arguments}
Pointer to character array first name, pointer to character array last name
\subsubsubsection{Return Value}
void
\subsubsubsection{Function Outline in Pseudocode}
Copy contents of internal member variable character arrays for first and last name into the
character arrays passed into the function.
\subsubsubsection{Tracability}
This function will partially fulfil requirement 2.2.2.5 of SOW by providing a default constructor for the
employee record class.


\subsubsection{Function:  setName()}
\subsubsubsection{Purpose}
This function allows a user to set the private employee first name and last name character arrays.
\subsubsubsection{Arguments}
Pointer to character array first name, pointer to character array last name
\subsubsubsection{Return Value}
void
\subsubsubsection{Function Outline in Pseudocode}
Copy contents of passed pointer to character arrays of first name and last name into internal member variable character arrays for first and last name.
\subsubsubsection{Tracability}
This function will partially fulfil requirement 2.2.2.5 of SOW by providing a default constructor for the
employee record class.

\subsubsection{Function:  getDept()}
\subsubsubsection{Purpose}
This function allows a user to get the internal member value for department id.
\subsubsubsection{Arguments}
int reference variable
\subsubsubsection{Return Value}
void
\subsubsubsection{Function Outline in Pseudocode}
Copy contents of internal member value department id into int variable referenced by the function argument.
\subsubsubsection{Tracability}
This function will partially fulfil requirement 2.2.2.6 of SOW by providing a default constructor for the
employee record class.

\subsubsection{Function:  setDept()}
\subsubsubsection{Purpose}
This function allows a user to set the employee department id.
\subsubsubsection{Arguments}
Int for department id
\subsubsubsection{Return Value}
void
\subsubsubsection{Function Outline in Pseudocode}
Set internal value for department id equal to the passed int value.
\subsubsubsection{Tracability}
This function will partially fulfil requirement 2.2.2.6 of SOW by providing a default constructor for the
employee record class.

\subsubsection{Function:  getSalary()}
\subsubsubsection{Purpose}
This pointer function allows a user to get the employee's salary.
\subsubsubsection{Arguments}
Pointer double
\subsubsubsection{Return Value}
void
\subsubsubsection{Function Outline in Pseudocode}
Copy contents of member variable for employee salary to the pointer variable passed as a function argument.
\subsubsubsection{Tracability}
This function will partially fulfil requirement 2.2.2.7 of SOW by providing a default constructor for the
employee record class.

\subsubsection{Function:  setSalary()}
\subsubsubsection{Purpose}
This function allows a user to set the employee's salary.
\subsubsubsection{Arguments}
double for salary
\subsubsubsection{Return Value}
void
\subsubsubsection{Function Outline in Pseudocode}
Set member variable for salary equal to the passed double variable.
\subsubsubsection{Tracability}
This function will partially fulfil requirement 2.2.2.7 of SOW by providing a default constructor for the
employee record class.

\subsubsection{Function:  printRecord()}
\subsubsubsection{Purpose}
This function prints all info for employee record to the screen.
\subsubsubsection{Arguments}
None
\subsubsubsection{Return Value}
void
\subsubsubsection{Function Outline in Pseudocode}
Print all internal variable values to screen.
\subsubsubsection{Tracability}
This function will fulfil requirement 2.2.2.8 of SOW by providing a default constructor for the
employee record class.

\end{document}
%%% Local Variables:
%%% mode: latex
%%% TeX-master: t
%%% End:
