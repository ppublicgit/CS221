\documentclass[12pt]{article}%

\usepackage{titlesec}
\usepackage{hyperref}

\titleclass{\subsubsubsection}{straight}[\subsection]

\newcounter{subsubsubsection}[subsubsection]
\renewcommand\thesubsubsubsection{\thesubsubsection.\arabic{subsubsubsection}}
\renewcommand\theparagraph{\thesubsubsubsection.\arabic{paragraph}} % optional; useful if paragraphs are to be numbered

\titleformat{\subsubsubsection}
  {\normalfont\normalsize\bfseries}{\thesubsubsubsection}{1em}{}
\titlespacing*{\subsubsubsection}
{0pt}{3.25ex plus 1ex minus .2ex}{1.5ex plus .2ex}

\makeatletter
\renewcommand\paragraph{\@startsection{paragraph}{5}{\z@}%
  {3.25ex \@plus1ex \@minus.2ex}%
  {-1em}%
  {\normalfont\normalsize\bfseries}}
\renewcommand\subparagraph{\@startsection{subparagraph}{6}{\parindent}%
  {3.25ex \@plus1ex \@minus .2ex}%
  {-1em}%
  {\normalfont\normalsize\bfseries}}
\def\toclevel@subsubsubsection{4}
\def\toclevel@paragraph{5}
\def\toclevel@paragraph{6}
\def\l@subsubsubsection{\@dottedtocline{4}{7em}{4em}}
\def\l@paragraph{\@dottedtocline{5}{10em}{5em}}
\def\l@subparagraph{\@dottedtocline{6}{14em}{6em}}
\makeatother

\setcounter{secnumdepth}{4}
\setcounter{tocdepth}{4}

\begin{document}

\begin{titlepage}
	\clearpage\thispagestyle{empty}
	\centering
	\vspace{2cm}

	% Titles
	%{\large  \par}
	%\vspace{4cm}
	{\Huge \textbf{Software Design Document \\
        for \\
        Employee Record Database with Customer List}} \\
	\vspace{1cm}
	{\large \textbf{Programming Assignment 2 \\
      February 20, 2020} \par}
	\vspace{4cm}
	{\normalsize Prepared By \\ % \\ specifies a new line
      Paul Abers \\
      pa0034@uah.edu \par}
	\vspace{2cm}

    \vspace{2cm}

	% Information about the University
	{\normalsize Prepared For \\
		Mr. James Williamson \\
		CS 221, Data Structures in C++ \\
        Computer Science Department \\
        University of Alabama in Huntsville \par}

      \vspace{2cm}

	\pagebreak

\end{titlepage}

\tableofcontents

\section{System Overview}
The purpose of this assignment is to provide a simple and easy way
to access an employee record database. The database has three parts,
the employee record class, a store class and a customer list class.
The employee record class must store an employee ID,
employee name, department, annual salary of each employee and a customer
list class. The customer list class stores a list of store classes. The store
class stores information on a single customer store.

\section{Referenced Documents}
Programming Assignment 1 Statement of Work. \\
Programming Assignment 2 Statement of Work.

\section{Architectural Design}
\subsection{Concept of Execution}
This program creates a class structure to store information for an
individual employee, a store and a list of stores. The employee class stores an
employee's first and last name, a unique employee ID, the employee's department ID, the employee's salary,
and a list of customer stores for the employee. The customer list class stores a list of customer stores.

A database manager will have access to public get and set methods of the class
in order to set the various attributes for the employee as well as get them later.
There is also a get customer list function.
There is also a default constructor that initializes the class as well as a constructor
that handles all inputs being included. A quick and easy print will also be provided for quickly
displaying all attributes of the class.
The customer list class will have functions
for adding a store, removing a store, getting a store based on store id and printing a store info.

\subsection{Abstract Data Type}
The employee record structure is implemented with a class structure separated in a cpp and header file.

\subsection{Code Outline}
This program will consist of the following files: EmployeeRecord.h, EmployeeRecord.cpp, Store.h, Store.cpp,
CustomerList.h and CustomerList.cpp.
\hfill\\
\hfill\\
EmployeeRecord Class
\underline{Private Attributes}:
\begin{itemize}
    \item m\_iEmployeeID -- int value for employee id
    \item m\_sLastName -- character array of length 32 for last name
    \item m\_sFirstName -- character array of length 32 for first name
    \item m\_iDeptId -- int for department id
    \item m\_dSalary -- double for employee's salary
    \item m\_pCustomerList -- pointer to customer list object
    \end{itemize}
    \hfill\\
\underline{Public Methods}:
\begin{itemize}
    \item EmployeeRecord() -- default constructor
    \item EmplyoeeRecord() -- initialization constructor
    \item getID() -- return int value of employee id
    \item setID() -- set employee id
    \item getName() -- copy employee's first and last name into pointers passed
    \item setName() -- set employee's first and last name to pointers passed
    \item getDept() -- get value of employee's department
    \item setDept() -- set value of employee's department
    \item getSalary() -- pointer function to get employee's salary
    \item setSalary() -- set employee's salary
    \item printRecord() -- prints to screen all data for employee's record
    \item getCustomerList() -- return the pointer to the employee record's customer list object
\end{itemize}
\hfill\\
\hfill\\
CustomerList Class
\underline{Private Attributes}:
\begin{itemize}
    \item m\_pHead -- pointer to first store in list
    \end{itemize}
    \hfill\\
\underline{Public Methods}:
\begin{itemize}
    \item CustomerList() -- constructor for store
    \item ~CustomerList() -- destructor for store
    \item addStore() -- add a store to list
    \item removeStore() -- remove a store from list
    \item getStore() -- access a store from list
    \item updateStore() -- update a store in list
    \item printStoresInfo() -- print info for each store in list
\end{itemize}


\section{Detailed Design}
\subsection{Source File: EmployeeRecord.h and EmployeeRecord.cpp}
\subsubsection{Function: EmployeeRecord()}
\subsubsubsection{Purpose}
This is the default constructor for the EmployeeRecord class.
\subsubsubsection{Arguments}
This default constructor takes no arguments.
\subsubsubsection{Return Value}
A constructor, therefore no value is returned.
\subsubsubsection{Function Outline in Pseudocode}
Set employee id to 0 \\
set last name to "" \\
set first name to "" \\
set department id to 0 \\
set salary to 0.0
\subsubsubsection{Tracability}
This function will fulfil requirement 2.2.2.1 of SOW

\subsubsection{Function: EmplyoeeRecord()}
\subsubsubsection{Purpose}
This is the optional constructor to set all values passed into function.
\subsubsubsection{Arguments}
int employee id, character array pointer first name, character array pointer last name, int for
department id, double for salary.
\subsubsubsection{Return Value}
None
\subsubsubsection{Function Outline in Pseudocode}
Set employee id to ID\\
copy passed character array for lName into m\_sLastName\\
copy passed character array for fName into m\_sFirstName\\
set department id to dept\\
set salary to sal.
\subsubsubsection{Tracability}
This function will fulfil requirement 2.2.2.2 of SOW

\subsubsection{Function: \textasciitilde EmplyoeeRecord()}
\subsubsubsection{Purpose}
This is the destructor for the employee record.
\subsubsubsection{Arguments}
None
\subsubsubsection{Return Value}
None
\subsubsubsection{Function Outline in Pseudocode}
Properly destruct the class. Clean up and deallocate memory initialized for pointers for the first and last name
character arrays.
\subsubsubsection{Tracability}
This function will fulfil requirement 2.2.2.3 of SOW

\subsubsection{Function:  getID()}
\subsubsubsection{Purpose}
This function allows a user to get the private employee ID.
\subsubsubsection{Arguments}
None
\subsubsubsection{Return Value}
Int value stored for employee id.
\subsubsubsection{Function Outline in Pseudocode}
Return value of member stored employee ID.
\subsubsubsection{Tracability}
This function will partially fulfil requirement 2.2.2.4 of SOW

\subsubsection{Function:  setID()}
\subsubsubsection{Purpose}
This function allows a user to set the private employee ID.
\subsubsubsection{Arguments}
Int value to set the member stored employee id to.
\subsubsubsection{Return Value}
Void
\subsubsubsection{Function Outline in Pseudocode}
Set internal member variable for employee id to passed integer value.
\subsubsubsection{Tracability}
This function will partially fulfil requirement 2.2.2.4 of SOW


\subsubsection{Function:  getName()}
\subsubsubsection{Purpose}
This function allows a user to get the private employee first and last names.
\subsubsubsection{Arguments}
Pointer to character array first name, pointer to character array last name
\subsubsubsection{Return Value}
void
\subsubsubsection{Function Outline in Pseudocode}
Copy contents of internal member variable character arrays for first and last name into the
character arrays passed into the function.
\subsubsubsection{Tracability}
This function will partially fulfil requirement 2.2.2.5 of SOW


\subsubsection{Function:  setName()}
\subsubsubsection{Purpose}
This function allows a user to set the private employee first name and last name character arrays.
\subsubsubsection{Arguments}
Pointer to character array first name, pointer to character array last name
\subsubsubsection{Return Value}
void
\subsubsubsection{Function Outline in Pseudocode}
copy passed character array for lName into m\_sLastName\\
copy passed character array for fName into m\_sFirstName\\
\subsubsubsection{Tracability}
This function will partially fulfil requirement 2.2.2.5 of SOW

\subsubsection{Function:  getDept()}
\subsubsubsection{Purpose}
This function allows a user to get the internal member value for department id.
\subsubsubsection{Arguments}
noneint reference variable
\subsubsubsection{Return Value}
void
\subsubsubsection{Function Outline in Pseudocode}
return m\_iDeptID
\subsubsubsection{Tracability}
This function will partially fulfil requirement 2.2.2.6 of SOW and 2.0.2.4 of SOW2

\subsubsection{Function:  setDept()}
\subsubsubsection{Purpose}
This function allows a user to set the employee department id.
\subsubsubsection{Arguments}
Int for department id
\subsubsubsection{Return Value}
void
\subsubsubsection{Function Outline in Pseudocode}
Set internal value for department id equal to the passed int value.
\subsubsubsection{Tracability}
This function will partially fulfil requirement 2.2.2.6 of SOW

\subsubsection{Function:  getSalary()}
\subsubsubsection{Purpose}
Get the employee's salary.
\subsubsubsection{Arguments}
none
\subsubsubsection{Return Value}
void
\subsubsubsection{Function Outline in Pseudocode}
return m\dSalary
\subsubsubsection{Tracability}
This function will partially fulfil requirement 2.2.2.7 of SOW

\subsubsection{Function:  setSalary()}
\subsubsubsection{Purpose}
This function allows a user to set the employee's salary.
\subsubsubsection{Arguments}
double for salary
\subsubsubsection{Return Value}
void
\subsubsubsection{Function Outline in Pseudocode}
Set member variable for salary equal to the passed double variable.
\subsubsubsection{Tracability}
This function will partially fulfil requirement 2.2.2.7 of SOW

\subsubsection{Function:  printRecord()}
\subsubsubsection{Purpose}
This function prints all info for employee record to the screen.
\subsubsubsection{Arguments}
None
\subsubsubsection{Return Value}
void
\subsubsubsection{Function Outline in Pseudocode}
Print header for employee record\\
print employee id\\
print employee first name\\
print employee last name\\
print employee department id\\
print employee salary
\subsubsubsection{Tracability}
This function will fulfil requirement 2.2.2.8 of SOW

\subsection{Source File: CustomerList.h and CustomerList.cpp}
\subsubsection{Function: CustomerList()}
\subsubsubsection{Purpose}
This is the default constructor for the CustomerList class.
\subsubsubsection{Arguments}
This default constructor takes no arguments.
\subsubsubsection{Return Value}
A constructor, therefore no value is returned.
\subsubsubsection{Function Outline in Pseudocode}
Initialize m\_pHead
\subsubsubsection{Tracability}
This function will fulfil requirement 2.2.2.1 of SOW by providing a default constructor for the
employee record class.

\subsubsection{Function: \textasciitilde CustomerList()}
\subsubsubsection{Purpose}
This is the default destructor for the CustomerList class.
\subsubsubsection{Arguments}
This default constructor takes no arguments.
\subsubsubsection{Return Value}
A constructor, therefore no value is returned.
\subsubsubsection{Function Outline in Pseudocode}
Start at m\_pHead \\
Loop over each Store object in linked list \\
delete each Store object
\subsubsubsection{Tracability}
This function was not specified in Statement of Work.

\subsubsection{Function: addStore()}
\subsubsubsection{Purpose}
Add a store to the customer list.
\subsubsubsection{Arguments}
A pointer to a store object.
\subsubsubsection{Return Value}
Bool to indicate success of insertion.
\subsubsubsection{Function Outline in Pseudocode}
Set value for m\_pNext of Store object equal to store object pointer argument \\
return true
\subsubsubsection{Tracability}
This function will fulfil requirement 2.0.4.2.1 of SOW 2.

\subsubsection{Function: removeStore()}
\subsubsubsection{Purpose}
Remove a store from the customer list.
\subsubsubsection{Arguments}
Integer for store with id ID to remove
\subsubsubsection{Return Value}
A pointer to a store object.
\subsubsubsection{Function Outline in Pseudocode}
Start at m\_pHead \\
Iterate through list until m\_pHead's next store's id equals input id. \\
Set temporary store variable equal to m\_pHead's next store \\
Set m\_pHead's next pointer equal to m\_pHead's next next pointer\\
Set temporary store variables next pointer equal to NULL\\
break loop iteration \\
return temporary store.
\subsubsubsection{Tracability}
This function will fulfil requirement 2.0.4.2.2 of SOW2

\subsubsection{Function: getStore()}
\subsubsubsection{Purpose}
Return a pointer to a store object with a given Id if in the list.
\subsubsubsection{Arguments}
Integer for store ID to get.
\subsubsubsection{Return Value}
a pointer to a store object if Id found, else NULL
\subsubsubsection{Function Outline in Pseudocode}
Set return pointer to Null \\
Start at m\_pHead \\
Iterate through list until m\_pHead's next store's id equals input id. \\
If m\_pHead's next store id equals input id, set ret equal to m\_phead next store, and break loop \\
return temporary store.
\subsubsubsection{Tracability}
This function will fulfil requirement 2.0.4.2.3 of SOW2

\subsubsection{Function: updateStore()}
\subsubsubsection{Purpose}
Update a stores value
\subsubsubsection{Arguments}
Integer for store Id to update, char array for name of store, char array for address of store,
char array for city of store, char aray for street of store, char array for zipcode of store.
\subsubsubsection{Return Value}
A boolean to indicate success of insertion.
\subsubsubsection{Function Outline in Pseudocode}
set success equal to false \\
Start at m\_pHead \\
Iterate through list until m\_pHead's next store's id equals input id. \\
\quad If m\_pHead's next store id equals input id, call all set functions for next store data with input args \\
\quad \quad set success equal to true \\
\quad \quad break loop \\
return success
\subsubsubsection{Tracability}
This function will fulfil requirement 2.0.4.2.4 of SOW2

\subsubsection{Function: printStoresInfo()}
\subsubsubsection{Purpose}
Print all store info for all stores in customer list
\subsubsubsection{Arguments}
No arguments
\subsubsubsection{Return Value}
void
\subsubsubsection{Function Outline in Pseudocode}
Start loop at m\_pHead \\
Loop over each store in customer list \\
\tab print store info \\
return\\
\subsubsubsection{Tracability}
This function will fulfil requirement 2.0.4.2.5 of SOW2

\end{document}
%%% Local Variables:
%%% mode: latex
%%% TeX-master: t
%%% End:
